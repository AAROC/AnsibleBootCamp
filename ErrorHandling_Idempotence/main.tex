% \pdfminorversion=4
\documentclass[12pt, xcolor={dvipsnames}]{beamer}
\usepackage{beamerthemeshadow}
\usepackage{listings}
\lstset{showstringspaces=false}
\newcommand\YAMLcolonstyle{\color{red}\mdseries}
\newcommand\YAMLkeystyle{\color{black}\bfseries}
\newcommand\YAMLvaluestyle{\color{blue}\mdseries}

\makeatletter

% here is a macro expanding to the name of the language
% (handy if you decide to change it further down the road)
\newcommand\language@yaml{yaml}

\expandafter\expandafter\expandafter\lstdefinelanguage
\expandafter{\language@yaml}
{
  keywords={True,False,null,y,n,yes,no},
  keywordstyle=\color{PineGreen}\bfseries,
  basicstyle=\YAMLkeystyle,                                 % assuming a key comes first
  sensitive=false,
  comment=[l]{\#},
  morecomment=[s]{/*}{*/},
  commentstyle=\color{purple}\ttfamily,
  stringstyle=\YAMLvaluestyle\ttfamily,
  moredelim=[l][\color{orange}]{\&},
  moredelim=[l][\color{magenta}]{*},
  moredelim=**[il][\YAMLcolonstyle{:}\YAMLvaluestyle]{:},   % switch to value style at :
  morestring=[b]',
  morestring=[b]",
  literate =    {---}{{\ProcessThreeDashes}}3
                {>}{{\textcolor{red}\textgreater}}1
                {|}{{\textcolor{red}\textbar}}1
                {\ -\ }{{\mdseries\ -\ }}3,
}

% switch to key style at EOL
\lst@AddToHook{EveryLine}{\ifx\lst@language\language@yaml\YAMLkeystyle\fi}
\makeatother

\newcommand\ProcessThreeDashes{\llap{\color{cyan}\mdseries-{-}-}}

\newenvironment{wideitemize}{\itemize\addtolength{\itemsep}{12pt}}{\enditemize}

\definecolor{links}{HTML}{2A1B81}
\hypersetup{colorlinks,linkcolor=,urlcolor=links}

\begin{document}
\title{Ansible in depth: Error handling and Idempotence}
\author{Marco Fargetta}
\institute{INFN Sezione di Catania}
\date{\today}

\frame{\titlepage}

%\frame{\frametitle{Table of contents}\tableofcontents}


\section{Error Handling}
\begin{frame}[t]{Standard Behaviour}
  \begin{wideitemize}
    \item Error based on commands and modules return code
    \item Task error will stop the playbook for the host
    \begin{itemize}
      \item Some modules could behave differently
    \end{itemize}
    \item In some cases failures require different approaches
  \end{wideitemize}
\end{frame}

\begin{frame}[t, fragile]{Stop the playbook}
  \begin{wideitemize}
    \item In case of failure the playbook should fail
    \begin{itemize}
      \item In case of connected services all the hosts must be configured
      at the same time
    \end{itemize}
    \item In this case the option \alert{\texttt{any\_errors\_fatal}} can be used
  \end{wideitemize}
  \begin{block}{Example}
    \begin{lstlisting}[language=yaml]
- hosts: somehosts
  any_errors_fatal: True
  roles:
    - myrole
    \end{lstlisting}
  \end{block}
\end{frame}

\begin{frame}[t, fragile]{Ignore errors}
  \begin{wideitemize}
    \item Some errors are not really relevant for the configuration
    \item It is possible to ignore errors with \alert{\texttt{ignore\_errors}}
  \end{wideitemize}
  \begin{block}{Example}
    \begin{lstlisting}[language=yaml, basicstyle=\scriptsize]
- name: this will not be counted as a failure
  command: /bin/false
  ignore_errors: yes
    \end{lstlisting}
  \end{block}
\end{frame}

\begin{frame}[t, fragile]{Override the results}
  \begin{wideitemize}
    \item Changes are triggered by the module or the executed commands
    \item Status change can be controlled with \alert{\texttt{changed\_when}}
  \end{wideitemize}
  \begin{block}{Example}
    \begin{lstlisting}[language=yaml, basicstyle=\scriptsize]
tasks:

  - shell: /usr/bin/billybass --mode="take me to the river"
    register: bass_result
    changed_when: "bass_result.rc != 2"

  # this will never report 'changed' status
  - shell: wall 'beep'
    changed_when: False
  \end{lstlisting}
  \end{block}
\end{frame}

\begin{frame}[t, fragile]{What is an error}
  \begin{wideitemize}
    \item Error conditions are not always identifiable in the exit
    status of a command or a module
    \item Error can be visible in the output
    \item It is possible to rise the error if some condition is verified
  \end{wideitemize}
  \begin{block}{Example}
    \begin{lstlisting}[language=yaml, basicstyle=\scriptsize]
- name: this command prints FAILED when it fails
  command: /usr/bin/example-command -x -y -z
  register: command_result
  failed_when: "'FAILED' in command_result.stderr"
    \end{lstlisting}
  \end{block}
\end{frame}

\begin{frame}[t]{Force the handlers}
  \begin{wideitemize}
    \item On error all the following tasks will not be executed
    \item Also notified task will not execute
    \item This can leave the host in an inconsistent state
    \item Notified handlers can execute if the \alert{\texttt{force\_handlers}} is specified
    \begin{itemize}
      \item It can be specified in the playbook as \texttt{force\_handlers: True}
      \item It can be provided in the command line as \texttt{--force-handlers}
      \item It can be specified in the configuration
    \end{itemize}
  \end{wideitemize}
\end{frame}


\section{Idempotence}

\begin{frame}[t]{What is it?}
  \begin{block}{Definition of Idempotent (source Oxford Dictionary)}
    \begin{quote}
      Denoting an element of a set which is unchanged in value when multiplied or otherwise operated on by itself
    \end{quote}
  \end{block}

  \begin{wideitemize}
    \item In Mathematics a function is idempotent if $f(x)=f(f(x))$
    \item What is idempotent in Ansible?
    \begin{itemize}
      \item The ability to execute playbook multiple times without changes in the hosts
    \end{itemize}
  \end{wideitemize}
\end{frame}

\begin{frame}[t]{Ansible idempotent playbooks}
  \begin{wideitemize}
    \item Ansible modules are idempotent
    \begin{wideitemize}
      \item Unfortunately some exceptions are present
      \item As an example mysql module was not idempotent
      \begin{itemize}
        \item It was more a limitation than a future and it could be fixed
      \end{itemize}
    \end{wideitemize}
    \item Run multiple times a playbook should not generate any status change after the first execution
  \end{wideitemize}
\end{frame}

\begin{frame}[t]{Idempotence test}
  \begin{wideitemize}
    \item Playbook should be tested for the Idempotence
    \item It is enough to execute multiple run of the same playbook
    \begin{itemize}
      \item It is possible to automatise with Jenkins and/or other continuous integration tools
    \end{itemize}
    \item Idempotent playbook are more safe for the management of the inventory
    \begin{itemize}
      \item It is easier to recognise inconsistent status requiring changes
    \end{itemize}
    \item Tasks are not executed if the corresponding status is correct
  \end{wideitemize}
\end{frame}

\begin{frame}[t]{Is everything idempotent}
  \begin{wideitemize}
    \item Modules are idempotent but they are not the only actions Ansible can perform on remote hosts
    \item Shell script and remote command are not idempotent because they are always executed if the task is requested
    \item Playbook can always be idempotent
  \end{wideitemize}
\end{frame}

\begin{frame}[t]{Make playbook idempotent}
  \begin{wideitemize}
    \item Tasks can have conditions to trigger the execution
    \item It is possible to retrieve facts from remote hosts
    \item It is possible to execute command without status change but providing some output
    \item \alert{Combining conditions with facts and output it is possible to create idempotent tasks from non idempotent commands}
  \end{wideitemize}
\end{frame}

\begin{frame}[t, fragile]{Example of idempotent tasks}
  \begin{lstlisting}[language=yaml, basicstyle=\scriptsize]
- name: Start ldap server
  service: name={{ ldap_service }} state=started enabled=yes
  sudo: yes
  tags:
    - LDAP

- name: Retrieve ldap server configuration
  command: ldapsearch -Y EXTERNAL -H ldapi:/// -b cn=config
  register: ldap_config
  changed_when: "False"
  sudo: yes
  tags:
    - LDAP

- name: Copy modules file LDIF
  copy: src=etc/openldap/slapd.d/{{ ldap_module_file }} dest=/tmp/{{ ldap_module_file }}
  sudo: yes
  when: '"memberof" not in ldap_config.stdout'
  tags:
    - LDAP
  \end{lstlisting}
\end{frame}
%--- Next Frame ---%

\begin{frame}[t]{Thanks}
  Examples and other materials in these slides are from the following sources:
  \begin{wideitemize}
    \item \href{http://docs.ansible.com/ansible/index.html}{Official Ansible documentation}
    \item \href{http://tylerturk.com/testing-ansible-idempotency/}{Tyler Turk's blog}
  \end{wideitemize}
\end{frame}

\end{document}
